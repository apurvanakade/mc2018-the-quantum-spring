% !TEX root = index.tex

%basic AMS packages
\documentclass[reqno,letter, 11pt,twoside]{article}
\usepackage{amsmath}
\usepackage{amsthm}
\usepackage{amssymb}
\usepackage{hyperref}
\hypersetup{
	colorlinks = true,
	linkcolor = blue,
	anchorcolor = blue,
	citecolor = blue,
	filecolor = blue,
	urlcolor = blue
}
\usepackage{epigraph}
\usepackage{mathpazo}
\usepackage{tcolorbox}
\usepackage[margin=1in,includehead,includefoot]{geometry}
\usepackage{fancyhdr}
  \pagestyle{fancy}
  \fancyhf{}
  \fancyhead[LO]{The Quantum Spring}
  \fancyhead[RE]{Apurva}
  \fancyhead[LE]{Mathcamp}
  \cfoot{\thepage}

\usepackage{graphicx}
  \graphicspath{ {images/} }
\usepackage{float}
\usepackage{subcaption}
\usepackage{color}
\usepackage{mdframed}
\usepackage{enumitem}
  \setlist[enumerate]{label=\emph{\alph*})}% global settings, for all lists
\usepackage{tikz}
\usepackage[all,cmtip]{xy}
\usepackage{multicol}
% \renewcommand{\thefootnote}{\fnsymbol{footnote}}

% \makeatletter
% \@addtoreset{footnote}{section}
% \makeatother

%for setting the equation number to sync with the theorem numbers
\numberwithin{equation}{section}
\newcommand{\hint}[1]{\footnote{\raggedleft\rotatebox{180}{Hint: #1\hfill}}}

%How does latex not have these?
\DeclareMathOperator{\Ad}{Ad}
\DeclareMathOperator{\ad}{ad}
\DeclareMathOperator{\tr}{tr}
\DeclareMathOperator{\Tr}{Tr}
\DeclareMathOperator{\Hom}{Hom}
\DeclareMathOperator{\maps}{Maps}
\DeclareMathOperator{\im}{im}
\DeclareMathOperator{\rank}{rank}
\DeclareMathOperator{\coker}{coker}
\DeclareMathOperator{\Exists}{\exists}
\DeclareMathOperator{\Forall}{\forall}
\DeclareMathOperator{\res}{Res}
\DeclareMathOperator{\mor}{Res}

%simple operators which can be pretty useful
\newcommand{\pr}[2][\:]{\frac{\partial #1}{\partial #2}}
\newcommand{\innerp}[2]{\langle #1, #2 \rangle}
\newcommand*\conj[1]{\overline{#1}}
\newcommand*\norm[1]{\lVert #1 \rVert}

\theoremstyle{plain}
\newtheorem{thm}{Theorem}[section]
\newtheorem{prop}[thm]{Proposition}
\newtheorem{lem}[thm]{Lemma}
\newtheorem{cor}[thm]{Corollary}


\theoremstyle{definition}
\newtheorem{definition}[thm]{Definition}
\newtheorem{example}[thm]{Example}
\newtheorem{remark}[thm]{Remark}
\newtheorem{ans}[thm]{Ans.}

\newcounter{q}
\newtheorem{question}[q]{Question.}
\definecolor{light-gray}{gray}{0.95}
\newenvironment{ques}
{
	\begin{tcolorbox}[colback=light-gray,arc=0pt,outer arc=0pt,boxrule=0.5pt]
	 \begin{question}
			}
			{
		\end{question}
	\end{tcolorbox}
}

%Real numbers, complex numbers, etc.
\newcommand{\R}{\mathbb{R}}
\newcommand{\C}{\mathbb{C}}
\newcommand{\Z}{\mathbb{Z}}
\newcommand{\Q}{\mathbb{Q}}
\newcommand{\F}{\mathbb{F}_2}
\newcommand{\U}{\mathcal{U}}
\newcommand{\V}{\mathcal{V}}
\renewcommand{\L}{\mathcal{L}}
\renewcommand{\P}{\mathcal{P}}
\newcommand{\B}{\mathcal{B}}
